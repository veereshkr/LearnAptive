\documentclass[titlepage]{article}
\title{Arithmetic Loops}
\usepackage{hyperref}
\hypersetup{colorlinks=true,urlcolor=blue}

\begin{document}
\begin{section}{Question1}
%			"Type":"MC1"\\
%			"Answer":2\\
%			"Difficulty":5\\
%			"Tag":["1.7.1.3.","1.7.1.2.","1.7.1.1.","1.7.2.1.","1.1.2.1.","1.1.1."]\\
The ratio of manufacturing cost of item A and item B is 3:5. The manufacturer sets the printed selling price of both the items 30\% higher than their respective manufacturing cost. If the Seller gives 10\% off on the printed prices and makes a profit of \$40.8 by selling both the items, what is the difference in manufacturing cost of item A and item B?
		\begin{description}
		\item[1]
			\$50
		\item[2]
			\$60
		\item[3]
			\$90
		\item[4]
			\$150
		\item[5]
			\$240
		\item[6]
			I don't understand the question
		\end{description}
\begin{subsection}{Question1.1}
%		"type":"MC1"\\
%		"Answer":3\\
%		"Difficulty":3
%		"Tag":["1.7.1.3.","1.7.1.2.","1.7.1.1.","1.1.2.1.","1.1.1."]\\
If the Printed price of an item is 30\% more than its production cost, and the selling price of the item is 10\% less than the printed price, how much profit did the seller make over the printed price?
		\begin{description}
		\item[1]
			40\%
		\item[2]
			20\%
		\item[3]
			17\%
		\item[4]
			13\%
		\item[5]
			10\%
		\item[6]
			I don't understand this question		
		\end{description}
\begin{subsubsection}{Question1.1.1}
%		"type":"SR1"\\
%		"Answer":\$130\\
%		"Difficulty":1
%		"Tag":["1.7.1.2.","1.7.1.1.","1.1.2.1.","1.1.1."]\\
If the production cost of an item is \$100 and the printed selling price of the item is 30\% more than the production cose, what is the printed selling price?
		\begin{description}
		\item[1]
			[textbox]
		\item[2]
			I don't understand this question
		\end{description}						
\end{subsubsection}
\begin{subsubsection}
%		"type":"SR1"\\
%		"Answer":3\\
%		"Difficulty":2
%		"Tag":["1.7.1.2.","1.7.1.1.","1.1.2.1.","1.1.1."]\\
If the Printed price of an item is \$130 and the seller sells the item at 10\% less than the printed price, what is the selling price of the item?
		\begin{description}
		\item[1]
			[textbox]
		\item[2]
			I don't understand this question	
		\end{description}		 
\end{subsubsection}
\end{subsection}
\begin{subsection}{Question1.2}
%		"type":"MC1"\\
%		"Answer":4\\
%		"Difficulty":2
%		"Tag":["1.7.1.1.","1.1.2.1."]\\
If 17\% of some amount is \$40.8, what is the total amount?
		\begin{description}
		\item[1]
			\$2.4
		\item[2]
			\$6.93
		\item[3]
			\$170
		\item[4]
			\$240
		\item[5]	
			\$693
		\item[6]
			I don't understnd the question
		\end{description}
\end{subsection}
\begin{subsection}{Question1.3}
%		"type":"MC1"\\
%		"Answer":5\\
%		"Difficulty":3
%		"Tag":["1.7.2.1.","1.1.2.1."]\\
If the ratio of cost of item A to the cost of item B is 3:5 and the total cost of the two item is \$240 then what is the cost of item A and item B respectively?
		\begin{description}
		\item[1]
			\$144, \$96
		\item[2]
			\$96, \$144
		\item[3]
			\$120, \$120
		\item[4]
			\$150, \$90
		\item[5]
			\$90,\$150
		\item[6]
			I don't understand this question
		\end{description}
\begin{subsubsection}{Question1.3.1}
%		"type":"MC1"\\
%		"Answer":3\\
%		"Difficulty":2
%		"Tag":["1.7.2.1."]\\
If the ratio of production cost of item A to that of item B is 3:5, what portion of the total production cost is the production cost of item A?
		\begin{description}
		\item[1]
			$\frac{3}{5}$
		\item[2]
			$\frac{2}{5}$
		\item[3]
			$\frac{3}{8}$
		\item[4]
			$\frac{5}{8}$
		\item[5]
			$\frac{2}{3}$
		\item[6]
			I don't understand this question
\end{description}
\end{subsubsection}
\begin{subsubsection}{Question1.3.2}
%		"type":SR1"\\
%		"Answer":3\\
%		"Difficulty":2
%		"Tag":["1.1.2.1."]\\
what is the $\frac{3}{8}$th of \$240?
		\begin{description}
		\item[1]
			[textbox]
		\item[2]
			I don't understand this question
		\end{description}		 
\end{subsubsection}
\end{subsection}
\end{section}
\begin{section}{Solution1}
Let us, for the time being, treat the two items together as a single item. Let 
\begin{subsection}{Solution1.1}
\begin{subsubsection}{Solution1.1.1}
the production cost of the item be \$100. Now, the printed price is 30\% more than the production cost. Therefore, the printed price is  
\begin{center}
$\$(100+100\times30\%)=\$130$.
\end{center}
\end{subsubsection}
\begin{subsubsection}{Solution1.1.2}
The selling price is 10\% less than the printed price. Now, 10\% of \$130 is \$13. Therefore the selling price is 
\begin{center}
\$130-\$13=\$117.
\end{center}
\end{subsubsection}
The production cost of the item is \$100 and the selling price is \$117. Hence seller made \$17 profit. \\
Now \$17 profit on \$100 production cost means the seller made 17\% profit.
\end{subsection}
\begin{subsection}{Solution1.2}

 17\% of some amount is \$40.8.\\
 1\% of the amount is $\frac{40.8}{17}.$\\
the whole amount (i.e. 100\%) is 
\begin{center}
\$$\frac{40.8}{17}\times100$ =\$$2.4\times100$=\$240
\end{center}
\end{subsection}
Therefore, The total production cost of the two items is \$240. Now,
\begin{subsection}{Solution1.3}
\begin{subsubsection}{Solution1.3.1}
The ratio of production cost of item A and item B is 3:5. Let the cost of Item A is \$$3x$. Then the cost of item B is \$$5x$. Total cost is \$$(3x+5x)=8x$. Hence, the production cost of item A is $\frac{3x}{8x}=\frac{3}{8}$th portion of the total cost.\\
\end{subsubsection}
Therefore cost of item A = 
\begin{subsubsection}{Solution1.3.2}
\$$240\times\frac{3}{8}$=\$90.
Therefore cost of item B is \$240-\$90=\$150
\end{subsubsection}
\end{subsection}
\end{section}
\begin{section}{Question2}

%"Answer" : 2 \\ 
%"Difficulty": 2 \\
%"Tag": ["1.6"., "1.6.1.1.", "1.6.1.2.1.", "1.6.1.2.2.", "1.6.1.2.3.", "1.6.1.2.4.", "1.6.1.2.5.", "1.6.1.3.1.", "1.6.1.3.2.", %"1.6.3.2.", "1.6.3.3."] \\
%"Action":{1:\hyperlink{Q2}{Q2},2:\hyperlink{Q3}{Q3},3:\hyperlink{Q3}{Q3},4:\hyperlink{Q3}{Q3},5:\hyperlink{Q3}{Q3}} \\


What is the greatest common factor of $315$ and $336$?
\begin{description}
\item[1] $18$

\item[2] $21$

\item[3] $31$

\item[4] $7$

\item[5] $1$

\item[6] I do not understand the question.

\end{description}


\begin{subsection}{Question2.1}
% answer = 2

 Which of the following is the prime factorization of 315? 
 \begin{description}
      \item[1] $2 \times 3^2 \times 5  \times 11$ \\
      \item[2] $3^2 \times 5 \times 7$\\
      \item[3] $2 \times 5^2 \times 7 $ \\
      \item[4] $2^5 \times 3^2 \times 5$ \\
      \item[5] $2^2 \times 3 \times 5^3$
      \item[6] I do not understand the question.
  \end{description}
  
  \begin{subsubsection}{Question2.1.1}
 %answer = 1,2,3
Which of the following divide $315$? (There can be more than one choices.)
\begin{description}
\item[1] $5$

\item[2] $7$

\item[3] $9$

\item[4] $10$

\item[5] $11$

\item[6] I do not understand the question.

\end{description}

\end{subsubsection}
\begin{subsubsection}{Question 1.1.2}
%answer = 4
What is $3^2 \times 5 \times 7$?
\begin{description}
\item[1] $565$

\item[2] $210$

\item[3] $185$

\item[4] $315$

\item[5] $435$

\item[6] I do not understand the question.
\end{description}


\end{subsubsection}
\end{subsection}
  
  \begin{subsection}{Question2.2}
%answer = 3
 Which of the following is the prime factorization of 336? 
 \begin{description}
      \item[1] $2^4 \times 5 \times 11 $ \\
      \item[2] $2 \time 3^3 \times 7 \times 5 $\\
      \item[3] $2^4 \times 3 \times 7 $ \\
      \item[4] $2 \times 5^3  \times 3^2 $ \\
      \item[5] $2^2 \times 3 \times 5^3 $
      \item[6] I do not understand the question.
  \end{description}


  

\begin{subsubsection}{Question2.2.1}
%answer = 1, 3, 5
Which of the following divide $336$? (There can be more than one choices.)
\begin{description}
\item[1] $3$

\item[2] $5$

\item[3] $7$

\item[4] $11$

\item[5] $16$

\item[6] I do not understand the question.

\end{description}

\end{subsubsection}
\begin{subsubsection}{Question2.2.2}
%answer = 2

What is $2^4 \times 3 \times 7$?
\begin{description}
\item[1] $565$

\item[2] $336$

\item[3] $180$

\item[4] $217$

\item[5] $433$

\item[6] I do not understand the question.
\end{description}

\end{subsubsection}
\end{subsection}

\end{section}

%%%%%%%%%%%%%%%%%% SOLUTION %%%%%%%%%%%%%%%%%%%

\begin{section}{Solution2}
%What is the greatest common factor of $3150$ and $3360$?
To obtain GCF of two numbers, the standard approach is to compute the prime factorizations of the given numbers and collect the largest common factors.
First we will obtain the prime factorization of $315$ then we will obtain the prime factorization
of $336$.
\begin{subsection}{Solution2.1}
To obtain the prime factorization of a number we guess its divisor
and then divide by that number and repeatedly find more divisors.


\begin{subsubsection}{Solution2.1.1}
\end{subsubsection}
Using basics of divisibility we can check that:
$5, 7$ and $9,$  divide $315$. 
%Expand on divisibility rules

\begin{subsubsection}{Solution2.1.2}
Using the basics of multiplication we can check that:

$3^2 \times 5 \times 7 = 9 \times 5 \times 7 = 9 \times 35 = 315.$
\end{subsubsection}
Thus $315 = 3^2 \times 5 \times 7$.
\end{subsection}
  
  \begin{subsection}{Solution2.2}
  To obtain the prime factorization of a number we guess its divisor
and then divide by that number and repeatedly find more divisors.

 
\begin{subsubsection}{Solution2.2.1}
Using basics of divisibility we can check that:
$6, 7,$ and $8$ divide $336$. 
%Expand on divisibility rules

\end{subsubsection}
\begin{subsubsection}{Solution2.2.2}
Using the basics of multiplication we can check that:

$2^4 \times 3 \times 7 = 16 \times 3 \times 7 = 48 \times 7 = 336.$

\end{subsubsection}
Thus: $336 = 2^4 \times 3 \times 7$.
\end{subsection}

Since $315 = 3^2 \times 5 \times 7$ and $336 = 2^4 \times 3 \times 7$,
the GCF of $315$ and $336$ is $3 \times 7 = 21$.

\end{section}

\end{document}

